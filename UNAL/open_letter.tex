\documentclass[11pt]{article}
\usepackage[spanish]{babel}
\usepackage[latin1]{inputenc}
\usepackage[pdftex]{color,graphicx}
\usepackage{draftwatermark}
\SetWatermarkText{DRAFT}
\SetWatermarkScale{10}
%\usepackage{garamond}
%\usepackage[T1]{fontenc}
\usepackage{hyperref}
\setlength{\oddsidemargin}{0cm}
\setlength{\textwidth}{450pt}
%\setlength{\topmargin}{-40pt}
%\addtolength{\hoffset}{-0.3cm}
%\addtolength{\textheight}{4cm}

\pagenumbering{arabic}
\bibliographystyle{unsrt}

\begin{document}
\large
Sirva la presente para solicitar muy respetuosamente a las autoridades
de la Universidad Nacional de Colombia, de la Facultad de Ciencias,
del Departamento de F\'isica, del Observatorio Astron\'omico Nacional y de
la Vicerrector\'ia de Investigaciones, tomar posici\'on al respecto a una
sospecha de plagio sistem\'atico por parte de Alexis Larra\~naga, profesor
del Observatorio Astron\'omico Nacional, en \textit{preprints} y publicaciones
del \'area de Gravitaci\'on y Cosmolog\'ia.  
\\

A continuaci\'on relacionamos cinco casos de conocimiento p\'ublico
que creemos deben ser considerados. 
\begin{enumerate}
\item En julio del 2015 apareci\'o publicado un aviso de retracci\'on de un
art\'iculo escrito por Alexis y dos estudiantes de la Universidad
Nacional.  El aviso cita que la raz\'on principal para retirar el
art\'iculo es plagio [1]. El mismo art\'iculo tambi\'en fue retirado por los
administradores del arXiv manifestando un excesivo reuso de texto de
otros autores [2]. Es de hacer notar que arXiv, administrado por la
Universidad de Cornell en los Estados Unidos es el mayor y m\'as
importante repositorio de publicaciones y \textit{preprints} en F\'isica,
Astrof\'isica y Matem\'aticas. 

\item En octubre del 2014 los administradores del arXiv marcaron un
  \textit{preprint} escrito por Alexis y dos estudiantes de la
  Universidad Nacional por tener un solapamiento significativo con
  un art\'iculo de otros autores (3). Casi dos semanas despu\'es,
  aparece una nueva versi\'on y casi un a\~no despu\'es vuelve a
  cambiar por una tercera versi\'on [4). Las versiones   recientes ya
    no presentan sobrelapamiento detectable por el arXiv.   

\item En abril del 2012 los administradores del arXiv marcaron un
  \textit{preprint} escrito por Alexis y una estudiante
  de la Universidad Nacional por reutilizar significativamente
  textos de otro art\'iculo del mismo profesor (4). Aparentemente esta
  misma versi\'on fue publicada en una revista.

\item En el mismo abril del 2012 otro \textit{preprint} de Alexis en el arXiv
  es marcado por tener un sobrelape en texto con un art\'iculo de otros
autores (5). El art\'iculo fue retirado del arXiv por el mismo Alexis
sin escribir ninguna explicaci\'on (6). Finalmente una nueva versi\'on del
\textit{preprint} aparece publicada en una revista. No tenemos datos p\'ublicos
sobre si existe sobrelapamiento de la versi\'on de la revista con otros
art\'iculos.  

\item De nuevo en abril del 2012 aparece otro \textit{preprint} de
  Alexis que fue marcado por tener un sobrelape en texto con un
  art\'iculo de otros autores (7). Aparentemente el \textit{preprint}
  nunca fue revisado para ser enviado a una revista. 
\end{enumerate}

En vista de los hechos anteriores solicitamos amablemente a los
directivos de la Universidad Nacional, a  su Facultad de Ciencias y a
su Vicerrector\'ia de Investigaciones que por favor aclaren de manera
p\'ublica si, para cada uno de estos cinco eventos:  
\begin{itemize}
\item ?` se ha llevado a cabo alguna investigaci\'on interna?
\item ?` se conoce el contexto en el que se han dado estos casos?
\item ?` se han buscando maneras de determinar en qu\'e grado la contribuci\'on intelectual de los \textit{preprints} y art\'iculos es original, m\'as all\'a de simples an\'alisis estad\'isticos del texto?
\item ?` se conoce el grado de responsabilidad del profesor y los estudiantes involucrados en la preparaci\'on de los \textit{preprints} y art\'iculos se\~nalados?
\end{itemize}

Aunque entendemos que no es su obligaci\'on hacer este tipo de
aclaraciones p\'ublicas, creemos que s\'i es un deber profesional y moral
con la comunidad acad\'emica nacional, inclu\'idos en primer lugar el
profesor y los estudiantes involucrados. 

Los abajos firmantes sentimos que la comunidad astron\'omica colombiana
pierde credibilidad cuando estos eventos suceden y adem\'as se dejan sin
ninguna explicaci\'on. En especial cuando se trata de la Universidad
p\'ublica m\'as respetada del pa\'is y, justo en estos momento cuando un
grupo de investigadores, entre ellos Alexis Larra\~naga, ingresan como
miembros de la Uni\'on Astron\'omica Internacional. 
 
Creemos que una aclaraci\'on p\'ublica de la parte de ustedes puede
llevarnos a una sana reflexi\'on sobre las din\'amicas de publicaci\'on e
investigaci\'on en Colombia, los diferentes controles de calidad que se
pueden llevar a cabo para evitar este tipo de situaciones y, sobre
todo, reforzar el valor de la transparencia. 


Firmantes

Referencias


\noindent
\end{document}

